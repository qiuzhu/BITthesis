\begin{zhabstract}
 流体喉部喷管技术是一项新兴的推力矢量控制技术,它相对于目前普遍在固体火箭发动机中采用的机械推力矢量控制技术方法的优点是不需要驱动侯栓等传动伺服机构,减小了结构尺寸及质量,且没有可移动部件,使得可靠性增强。流体喉部喷管技术一般是通过二次流体喷射,使主流与二次流发生相互作用,使得主流的喉部流通面积和喉部形状发生改变。
 
 固体火箭发动机燃烧室内不仅仅存在多种引起声能的增益因素,同时也存在各种
 阻尼因素,它们能够造成声能损失,使得压力振荡逐渐衰减。固体火箭发动机中比较重要的阻尼因素有喷管阻尼、微粒阻尼、壁面阻尼、结构阻尼、头部空腔阻尼等,其中喷管阻尼对轴向声不稳定燃烧有很强的阻尼作用。由于喷管进口截面恰好位于声压波腹位置,此处平均气流速度较高,所以声能辐射和对流损失都比较大,在固体火箭发动机内,喷管带来的声能损失常占总声能损失的一半以上。因此,深入开展喷管阻尼特性研究对固体火箭发动机不稳定燃烧预估及抑制有着重要的意义。
 
 目前国内外对于普通喷管已经开展了比较多的阻尼实验与仿真研究,而流体喉部技术二次流的施加一定会对喷管的阻尼有所影响,目前尚没有针对于施加了二次射流的喷管的阻尼的专门研究。为了探究流体喉部二次流对喷管阻尼的影响,文章通过实验和仿真的角度,利用稳态波衰减法以及脉冲衰减法的原理,得到了不同工况下流体喉部喷管的阻尼特性;主要讨论了不同的二次流喷射角度以及不同的主流二次流流量比下的阻尼特性变化规律,对比实验仿真结果,相互验证结论可靠性,进一步完善了流体喉部喷管推力矢量控制技术的内容,为这项技术的实际应用提供了参考。
 
\end{zhabstract}

\zhkeywords{固体火箭发动机;流体喉部;喷管阻尼}

\begin{enabstract}
 Fluid throat nozzle technology is a new thrust vector control technology, which is generally used in solid rocket motors used in mechanical thrust vector control technology has the advantage of not driving the drive, such as Hou bolt, etc., reducing the Structure size and quality, and no moving parts, making the reliability enhancement. Fluid throat nozzle technology is generally through the secondary fluid injection, the mainstream and secondary flow interaction, making the mainstream of the throat flow area and throat shape changes.
 
 Solid rocket motor combustion chamber not only exist a variety of factors that cause sound energy gain, but also there are various Damping factors, which can cause loss of sound energy, so that the gradual decay of pressure oscillation. The most important damping factors in solid rocket motors are nozzle damping, particle damping, wall damping, structural damping, and head cavity damping, among which nozzle damping has a strong damping effect on axial acoustic instability combustion. Since the inlet section of the nozzle is located at the position of the acoustic pressure antinode, where the average air velocity is high, the sound energy radiation and convection loss are relatively large. In the solid rocket engine, the sound energy loss caused by the nozzle often accounts for the total sound Can lose more than half. Therefore, it is of great significance to study the damping characteristics of the nozzle in order to predict and suppress the unstable combustion of the solid rocket motor.
 
 At present, more and more damping experiments and simulations have been carried out for ordinary nozzles at home and abroad. The application of secondary flow of fluid throat technology will certainly affect the damping of the nozzle. At present, there is no research on the application of secondary jet The damping of the nozzle is specialized. In order to investigate the effect of the secondary flow of the fluid throat on the nozzle damping, the damping characteristics of the throat nozzle under different operating conditions were obtained by the steady-state wave attenuation method and the pulse attenuation method. The results of the comparison of the simulation results and the reliability of the mutual verification conclusion further improve the content of the thrust throat nozzle thrust vector control technology, which provides a reference for the practical application of this technology.
 
\end{enabstract}

\enkeywords{Solid rocket engine; Fluid throat; Secondary flow; catalytic experiment; The oxidation experiment}
