
\documentclass[oneside,master]{BITthesis}

%% openany,|openright,可选,双面打印时每章的第一页仅右页开启,默认右页开启(openright)
%% oneside|twoside ,可选,默认开启oneside
%% master|doctor,必需,博士采用doctor选项,默认选项master为硕士

% 请在bitthesisextra.sty文件中定义其他会用到的宏包和自己的变量
% 这样可以防止main.tex太过臃肿。
\usepackage{bitthesisextra}
% =====采用biblatex方式,需添加文献库===========
\addbibresource{reference/test.bib}
%%-----------以下是封面的填入栏目,按照格式填入即可================

\zhtitle{BITthesis 研究生论文模板说明和一些使用技巧}
\entitle{BITthesis manual and some tips about \LaTeXe{}}
\zhauthor{北理联盟}
\enauthor{lotus}
\zhadvisor{***}
\enadvisor{David}
\zhchairman{哈教授}
\enchairman{Dr.Wanfg}
\zhdegree{工程硕士}
\endegree{Master of Engineering}
\zhdate{2017年7月7号}
\endate{2017}
\zhacademy{宇航学院}
\enacademy{Aerospace Engineering}
\zhmajor{航空宇航科学与技术}
\enmajor{Aerospce tecnology}
\zhschool{北京理工大学}
\enschool{Beijing Institute of Technology}
\classified{TQ028.1}
\udc{540}


\begin{document}

\makecover     %%% 封面部分

\begin{zhabstract}
 临近毕设,学院里提供了毕业设计的 \LaTeX{}模板意见征求稿,于是决定离开简单的 Markdown ,开始使用 \LaTeX{} 进行写作。但是发现研究生院提供的模板格式存在很大的问题,编译也出现错误,于是修改重新设计了模板,欢迎再欢迎提交建议和意见,欢迎高质量的PR。项目地址为\url{https://github.com/qiuzhu/BITthesis}
 
\end{zhabstract}

\zhkeywords{模板;北京理工大学;毕业设计}

\begin{enabstract}
 Fluid throat nozzle technology is a new thrust vector control technology, which is generally used in solid rocket motors used in mechanical thrust vector control technology has the advantage of not driving the drive, such as Hou bolt, etc., reducing the Structure size and quality, and no moving parts, making the reliability enhancement. Fluid throat nozzle technology is generally through the secondary fluid injection, the mainstream and secondary flow interaction, making the mainstream of the throat flow area and throat shape changes.
 
 
\end{enabstract}

\enkeywords{template; \LaTeX; Github; experiment; BIT}

%% 符号对照表,可选,如不用可注释掉
\input{chapters/denotation}
{\hypersetup{linkcolor=black}    %%此处设置目录选项的链接为黑色,仅仅影响本地此刻的范围
    \tableofcontents
%% 插图、表格索引,可选,如不用可注释掉
    \listoffigures
    \listoftables
}
\mainmatter  %%% 主体部分(绪论开始,结论为止)
%%======================================================================%%
%* 子文件的多少和内容由你决定(最好以章为单位),基本原则是提速预览、脉络清晰、管理容易。
%%======================================================================%%


\chapter{\LaTeX{}介绍}
\section{使用\LaTeX{}的写毕业论文的有点}
有人还写过论文,参见\url{http://journals.plos.org/plosone/article?id=10.1371/journal.pone.0115069}
在我看来,最大的优点在于:
\begin{itemize}
	\item 数学公式的自动编号和交叉引用\cite{merry_1975_PenetrationVerticalJets}
	\item 文件干净,随手记事本或者Vim或者nano都能编辑,不像Word的docx解压以后一堆人眼无法阅读的xml文档\cite{mendez_1998_StudyGasVelocity}
	\item 因为文件干净,自动化也很方便,Bash、Python……都可以干活(当然Word也可以通过VBS和C进行很强大的自动化)
	\item 强迫用户以结构化的方式写作,输出的PDF结构树清晰\cite{crawford_1995_FutureLibrariesDreams}
	而Word默认导出PDF是不输出结构的,需要另外勾选,当然如果勾选了的话不比LaTeX差[附图1]
	\item 各种各样的宏包,TikZ这种包估计Word万年都不会有对应的插件\cite{allen_1996_ReviewPerformanceEngineering,bilitza_2014_InternationalReferenceIonosphere,bilitza_2012_MeasurementsIRIModel,bilitza_2008_InternationalReferenceIonosphere,zhanghui_2012_WoGuoCuiHuaLieHuaGongYiJiZhuFaZhanYuQuShi}
	\item 模板质量都很高,各种边距都考虑得很周到,而且切换方便,可以管理的格式很多,如[1]中提到的分栏问题Word的模板是解决不了的,因为本质上Word里“分栏”是页面的属性而不是段落的属性UNIX-friendly
	\item 长度单位不依赖于系统的地区设置各种特殊页面界定清晰,修改灵活,不像Word的“封面”功能有些莫名奇妙~
	\item 矢量图只要用了合适的包和编译引擎就能支持很多格式,不像Word只支持emf或者wmf题注系统比Word强到不知道哪里去了
    \item Computer Modern系列字体是真的美,美出声。
\end{itemize}


\section{公式举例}
附录有跟多类型的公式,可以观摩此公式编辑功能的美:

线性稳定性主要研究小扰动振幅的变化规律,主要基础为燃烧室内的一维波动方程。燃烧室内压力$P$可表示为:
\begin{equation}
P=\dot{p}+p_0 e^{\alpha t} e^{j(\omega t+hx)}\label{eq:pressue p}
\end{equation}
若$\alpha>0$,小扰动有增长趋势,则燃烧不稳定;若$\alpha>0$,则小扰动有减弱趋势,燃烧具有稳定性,其增长常数α可表示为各种增益、阻尼效果之和,如式\ref{eq:zengyi}所示。
\begin{equation}
\alpha=\alpha_{pc}+\alpha_{vc}+\alpha_{dc}+\alpha_n+\alpha_p+\alpha_{mf}+\alpha_{g}+\alpha_{w}+\alpha_{st}\label{eq:zengyi}
\end{equation}

\section{一些题外话}
\subsection{内容为王}
论文当然是内容为王,不应该是被格式分心
\begin{itemize}
	\item 不会用LaTeX --> 无法编译 没有文档
\item	不会用word --> 文档真难看 格式丑死了
	
\item 	会用LaTeX --> 漂亮的文档
\item	会用word --> 文档
	
\item	LaTeX 用的好 --> 牛逼的文档
\item	Word 用的好 --> 牛逼的文档
\end{itemize}

\subsection{word VS \LaTeX{}}
能用LaTeX的人,通常知道如何正确地使用LaTeX;能用Word的人,大多数根本就不会正确地使用Word,比如样式模板、“内容和样式分开管理”、域代码、VBA……而且上面好多人说的LaTeX可以直接套现成的模板……那是模板的功劳,幸好本文编写好了北理工的研究生模板啦!

总之:
\begin{itemize}
	\item word是开始觉得容易,后来觉得难,并且发现越来越难
	\item latex是开始觉得难,后来觉得容易,往后又发现难而且非常难,所以就凑合着用了,好在模板很多
\end{itemize}

\section{本文主要研究内容}
关于此模板的使用


  %%第一章内容
\chapter{喷管阻尼计算模型及原理}
\section{喷管阻尼}
喷管阻尼损失实质上是一种声场与平均流之间的相互作用。超声速喷管对燃烧室声场不显示开口特性,其原因在于喷管入口段具有很大的压强梯度与温度梯度,即构成了很多声学特性不同的横截面,有效地反射了声波。大部分声波能够被反射,小部分声能可以透过喷管以辐射形式传递到外界。此外,从流出喷管的燃气还能以对流形式带走一部分声能,这两部分声能的耗散将有效地衰减燃烧室内的声能。对于横向振型而言,喷管的阻尼作用非常小,它仅相当于刚性边界条件。因此,喷管阻尼主要是针对轴向振型而言。

喷管阻尼理论是根据一定简化模型提出的,只能用来预计收敛段光滑过度的喷管。若喷管收敛段几何形状复杂,则理论并不能作出确切的预计,必须通过试验或数值计算来测定其阻尼系数。

\section{稳态波衰减法和脉冲衰减法}
目前已经发展了多种冷流模拟试验,有直接法[198]、稳态波衰减法[199]、频率响应法[200]、脉冲法[7]和阻抗管法[201]。冷气状态与真实发动机工作状态之间的区别是介质温度和成分不同,因而平均声速及特征声阻抗也不同,因此,需要将冷态试验数据做进一步转换才能去表征真实发动机的喷管阻尼系数。通过试验能够测量不同复杂结构的喷管阻尼特性,但是试验方法成本较高而且试验周期较长。近年来,研究人员通过数值方法对喷管阻尼特性进行了计算分析[202][203][204],数值计算结果能够很好地吻合试验结果,为喷管阻尼研究提供了便利的数值工具。
本节将重点介绍稳态波衰减法及脉冲衰减法测量喷管阻尼常数的试验原理,通过借鉴试验原理,提出相应的数值计算方法。采用稳态波衰减法测量喷管阻尼时,首先需要调节声源,使其频率等于模拟燃烧室中的某阶固有频率,待燃烧内建立稳定的驻波以后,突然切断声源,当声源切断后,燃烧室内的压力将以指数形式衰减,通过动态压力监测系统记录测量点的瞬态压力曲线,计算压力衰减系数,计算所得衰减系数即为喷管阻尼,

稳态波衰减法原理是燃烧室空腔内施加一个稳定的、周期性的正弦压力波动信号,待发动机内建立起稳定的波形以后,突然切断稳定的压力波动源,让燃烧室内压力自动衰减,通过记录不同点的瞬态压力曲线,即可得到压力衰减速率。
压力振荡幅值与初始压强幅值之间的关系如下式所示:
\begin{equation}
P=p_0 e^{\alpha t}\label{eq:yaqiang}
\end{equation}
其中,为发动机内的衰减系数。在本研究中,可视为壁面阻尼,气体粘性损失,喷管阻尼等因素的总合,是发动机内的整体阻尼。需要说明的是,由于数值计算仅仅为纯流动仿真,最后所得的衰减系数与理论值会有较大的区别,但是,该方法可用来横向比较不同结构发动机内的阻尼特性,为优化发动机设计提供一定的工程指导。

可利用下式计算阻尼值:
\begin{equation}
\alpha=\dfrac{lnp_2-lnp_1}{t_2-t_1}\label{eq:zuning}
\end{equation}
  %%第二章内容
\include{chapters/chapter3}
\chapter{\LaTeX{}使用教程}

\section{字体命令}\label{txt:FreqCmd}
{\kaishu 玲珑骰子安红豆,入骨相思知不知。\hfill ——温庭筠}
	
{\fangsong 愿得一心人,白头不相离。\hfill ——卓文君}
		
{\ifcsname youyuan\endcsname\youyuan\else[无 \cs{youyuan} 字体。]\fi 去年今日此门中,人面桃花相映红。\hfill ——崔护}
			
{\heiti 入我相思门,知我相思苦。\hfill ——李白}
				
{\ifcsname lishu\endcsname\lishu\else[无 \cs{lishu} 字体。]\fi 此情可待成追忆?只是当时已惘然。\hfill ——李商隐}
					
{\songti 雨打梨花深闭门,忘了青春,误了青春。\hfill ——唐寅}

使用\cs{textbf}和\cs{textit}以及\cs{underline}的效果分别如下:

这句话的\textbf{文字}分别\textit{使用}了三种命令来\underline{处理}。

The \textbf{words} in this sentences are \textit{processed} with three different \underline{cmd}.

\section{表格样本}

\subsection{基本表格}
\label{sec:basictable}

模板中关于表格的宏包有三个: \pkg{booktabs}、\pkg{array} 和\pkg{longtabular}。三线表可以用 \pkg{booktabs}提供的 \cs{toprule}、\cs{midrule} 和 \cs{bottomrule}。它们与\pkg{longtable} 能很好的配合使用。
\begin{table}[htb]
	\centering
	\begin{minipage}[t]{0.9\linewidth} % 如果想在表格中使用脚注,minipage是个不错的办法
	\caption[模板文件]{模板文件。如果表格的标题很长,那么在表格索引中就会很不美观,所以要像 chapter 那样在前面用中括号写一个简短的标题。这个标题会出现在索引中。}
	\label{tab:template-files}
	\begin{tabularx}{\linewidth}{lX}
		\toprule
		{\heiti 文件名} & {\heiti 描述} \\
		\midrule
		cquthesis.cls & 模板类文件\footnote{这是一个脚注}\\
		cquthesis.cfg & 模板配置文件\footnote{这是又一个脚注}\\
		cqunumberical.bst & 参考文献 BIB\TeX\ 样式文件。\\
		cquthesis.sty & 常用的包和命令写在这里,减轻主文件的负担。\footnote{同一页上的脚注最多支持到10个}\\
		\bottomrule
		\end{tabularx}
	\end{minipage}
\end{table}

首先来看一个最简单的表格。\ref{tab:template-files} 列举了本模板主要文件及其功能。请大家注意三线表中各条线对应的命令。这个例子还展示了如何在表格中正确使用脚注。由于 \LaTeX{} 本身不支持在表格中使用\cs{footnote},所以我们不得不将表格放在小页中,而且最好将表格的宽度设置为小页的宽度,这样脚注看起来才更美观。

\begin{table}[htbp]
	\noindent\begin{minipage}{0.5\textwidth}
		\centering
		\caption{第一个并排子表格}
		\label{tab:parallel1}
		\begin{tabular}{p{2cm}p{2cm}}
					\toprule
					No. & Name \\\midrule
					1 & Fox \\
					2 & Panda \\
					3 & Dog \\
					\bottomrule
		\end{tabular}
	\end{minipage}%
	\begin{minipage}{0.5\textwidth}
		\centering
		\caption{第二个并排子表格}{The second subtable in one row}
		\label{tab:parallel2}
		\begin{tabular}{p{2cm}p{2cm}}
			\toprule
			No. & Name \\\midrule
			1 & Charlie \\
			2 & Jack \\
			3 & Tom \\
			\bottomrule
		\end{tabular}
	\end{minipage}
\end{table}

\begin{table}[htbp]
	\centering
	\caption{并排子表格}
	\label{tab:subtable}

	{
		\begin{tabular}{p{2cm}p{2cm}}
			\toprule
			111 & 222 \\\midrule
			222 & 333 \\\bottomrule
		\end{tabular}
	}
	\hskip2cm

	{
		\begin{tabular}{p{2cm}p{2cm}}
			\toprule
			111 & 222 \\\midrule
			222 & 333 \\\bottomrule
		\end{tabular}
	}
\end{table}

不可否认 \LaTeX{} 的表格功能没有想象中的那么强大,不过只要足够认真,足够细致,同样可以排出来非常复杂非常漂亮的表格。

\ref{tab:parallel1}和\ref{tab:parallel2}展示了\cs{xuhao}和\cs{xuhao}\texttt{[1]}的使用,可以达到自动编号的效果。不过要记得在每次使用之前使用\cs{resetxuhao},或者\cs{xuhao}\texttt{[1]}。使用\cs{setxuhao}\oarg{1-6}可以更改序号的标记方式,如\ref{tab:parallel2}所示。详细用法请参阅用户手册。

\begin{longtable}[c]{c*{6}{r}}
	\caption[实验数据]{实验数据,这个题注是双语的,而且十分的长,注意这在索引中的处理方式}\label{tab:performance}\\
	\toprule
	测试程序 & \multicolumn{1}{c}{正常运行} & \multicolumn{1}{c}{同步} & \multicolumn{1}{c}{检查点} & \multicolumn{1}{c}{卷回恢复}
	& \multicolumn{1}{c}{进程迁移} & \multicolumn{1}{c}{检查点} \\
	& \multicolumn{1}{c}{时间 (s)}& \multicolumn{1}{c}{时间 (s)}&
	\multicolumn{1}{c}{时间 (s)}& \multicolumn{1}{c}{时间 (s)}& \multicolumn{1}{c}{
		时间 (s)}&  文件(KB)\\\midrule
	\endfirsthead
	\multicolumn{7}{c}{续表~\thetable\hskip1em 实验数据}\\
	\toprule
	测试程序 & \multicolumn{1}{c}{正常运行} & \multicolumn{1}{c}{同步} & \multicolumn{1}{c}{检查点} & \multicolumn{1}{c}{卷回恢复}
	& \multicolumn{1}{c}{进程迁移} & \multicolumn{1}{c}{检查点} \\
	& \multicolumn{1}{c}{时间 (s)}& \multicolumn{1}{c}{时间 (s)}&
	\multicolumn{1}{c}{时间 (s)}& \multicolumn{1}{c}{时间 (s)}& \multicolumn{1}{c}{
		时间 (s)}&  文件(KB)\\\midrule
	\endhead
	\hline
	\multicolumn{7}{r}{续下页}
	\endfoot
	\endlastfoot
	CG.A.2 & 23.05 & 0.002 & 0.116 & 0.035 & 0.589 & 32491 \\
	CG.A.4 & 15.06 & 0.003 & 0.067 & 0.021 & 0.351 & 18211 \\
	CG.A.8 & 13.38 & 0.004 & 0.072 & 0.023 & 0.210 & 9890 \\
	CG.B.2 & 867.45 & 0.002 & 0.864 & 0.232 & 3.256 & 228562 \\
	CG.B.4 & 501.61 & 0.003 & 0.438 & 0.136 & 2.075 & 123862 \\
	CG.B.8 & 384.65 & 0.004 & 0.457 & 0.108 & 1.235 & 63777 \\
	MG.A.2 & 112.27 & 0.002 & 0.846 & 0.237 & 3.930 & 236473 \\
	MG.A.4 & 59.84 & 0.003 & 0.442 & 0.128 & 2.070 & 123875 \\
	MG.A.8 & 31.38 & 0.003 & 0.476 & 0.114 & 1.041 & 60627 \\
	MG.B.2 & 526.28 & 0.002 & 0.821 & 0.238 & 4.176 & 236635 \\
	MG.B.4 & 280.11 & 0.003 & 0.432 & 0.130 & 1.706 & 123793 \\
	MG.B.8 & 148.29 & 0.003 & 0.442 & 0.116 & 0.893 & 60600 \\
	LU.A.2 & 2116.54 & 0.002 & 0.110 & 0.030 & 0.532 & 28754 \\
	LU.A.4 & 1102.50 & 0.002 & 0.069 & 0.017 & 0.255 & 14915 \\
	LU.A.8 & 574.47 & 0.003 & 0.067 & 0.016 & 0.192 & 8655 \\
	LU.B.2 & 9712.87 & 0.002 & 0.357 & 0.104 & 1.734 & 101975 \\
	LU.B.4 & 4757.80 & 0.003 & 0.190 & 0.056 & 0.808 & 53522 \\
	LU.B.8 & 2444.05 & 0.004 & 0.222 & 0.057 & 0.548 & 30134 \\
	CG.B.2 & 867.45 & 0.002 & 0.864 & 0.232 & 3.256 & 228562 \\
	CG.B.4 & 501.61 & 0.003 & 0.438 & 0.136 & 2.075 & 123862 \\
	CG.B.8 & 384.65 & 0.004 & 0.457 & 0.108 & 1.235 & 63777 \\
	MG.A.2 & 112.27 & 0.002 & 0.846 & 0.237 & 3.930 & 236473 \\
	MG.A.4 & 59.84 & 0.003 & 0.442 & 0.128 & 2.070 & 123875 \\
	MG.A.8 & 31.38 & 0.003 & 0.476 & 0.114 & 1.041 & 60627 \\
	MG.B.2 & 526.28 & 0.002 & 0.821 & 0.238 & 4.176 & 236635 \\
	MG.B.4 & 280.11 & 0.003 & 0.432 & 0.130 & 1.706 & 123793 \\
	MG.B.8 & 148.29 & 0.003 & 0.442 & 0.116 & 0.893 & 60600 \\
	LU.A.2 & 2116.54 & 0.002 & 0.110 & 0.030 & 0.532 & 28754 \\
	LU.A.4 & 1102.50 & 0.002 & 0.069 & 0.017 & 0.255 & 14915 \\
	LU.A.8 & 574.47 & 0.003 & 0.067 & 0.016 & 0.192 & 8655 \\
	LU.B.2 & 9712.87 & 0.002 & 0.357 & 0.104 & 1.734 & 101975 \\
	LU.B.4 & 4757.80 & 0.003 & 0.190 & 0.056 & 0.808 & 53522 \\
	LU.B.8 & 2444.05 & 0.004 & 0.222 & 0.057 & 0.548 & 30134 \\
	EP.A.2 & 123.81 & 0.002 & 0.010 & 0.003 & 0.074 & 1834 \\
	EP.A.4 & 61.92 & 0.003 & 0.011 & 0.004 & 0.073 & 1743 \\
	EP.A.8 & 31.06 & 0.004 & 0.017 & 0.005 & 0.073 & 1661 \\
	EP.B.2 & 495.49 & 0.001 & 0.009 & 0.003 & 0.196 & 2011 \\
	EP.B.4 & 247.69 & 0.002 & 0.012 & 0.004 & 0.122 & 1663 \\
	EP.B.8 & 126.74 & 0.003 & 0.017 & 0.005 & 0.083 & 1656 \\
	\bottomrule
\end{longtable}

如果你要排版的表格长度超过一页,那么推荐使用 \pkg{longtable} 或者 \pkg{supertabular}宏包,模板对 \pkg{longtable} 进行了相应的设置,所以用起来可能简单一些。表~\ref{tab:performance} 就是 \pkg{longtable} 的简单示例。


\section{数学公式}
\label{sec:equation}
贝叶斯公式如式~(\ref{equ:chap1:bayes}),其中 $p(y|\mathbf{x})$ 为后验;
$p(\mathbf{x})$ 为先验;分母 $p(\mathbf{x})$ 为归一化因子。
\begin{equation}
\label{equ:chap1:bayes}
p(y|\mathbf{x}) = \frac{p(\mathbf{x},y)}{p(\mathbf{x})}=
\frac{p(\mathbf{x}|y)p(y)}{p(\mathbf{x})} 
\end{equation}

论文里面公式越多,\TeX{} 就越 happy。再看一个 \pkg{amsmath} 的例子:
\newcommand{\envert}[1]{\left\lvert#1\right\rvert} 
\begin{equation}\label{detK2}
\det\mathbf{K}(t=1,t_1,\dots,t_n)=\sum_{I\in\mathbf{n}}(-1)^{\envert{I}}
\prod_{i\in I}t_i\prod_{j\in I}(D_j+\lambda_jt_j)\det\mathbf{A}
^{(\lambda)}(\overline{I}|\overline{I})=0.
\end{equation} 

前面定理示例部分列举了很多公式环境,可以说把常见的情况都覆盖了,大家在写公式的时候一定要好好看 \pkg{amsmath} 的文档,并参考模板中的用法:
\begin{multline*}%\tag{[b]} % 这个出现在索引中的
\int_a^b\biggl\{\int_a^b[f(x)^2g(y)^2+f(y)^2g(x)^2]
-2f(x)g(x)f(y)g(y)\,dx\biggr\}\,dy \\
=\int_a^b\biggl\{g(y)^2\int_a^bf^2+f(y)^2
\int_a^b g^2-2f(y)g(y)\int_a^b fg\biggr\}\,dy
\end{multline*}

这里还有一个多级规划公式,这个公式使用\csgo{listeq}{索引名}手动加入了目录后的索引。
\begin{equation}\label{bilevel}
\left\{\begin{array}{l}
\max\limits_{{\mbox{\footnotesize\boldmath $x$}}} F(x,y_1^*,y_2^*,\cdots,y_m^*)\\[0.2cm]
\mbox{subject to:}\\[0.1cm]
\qquad G(x)\le 0\\[0.1cm]
\qquad(y_1^*,y_2^*,\cdots,y_m^*)\mbox{ solves problems }(i=1,2,\cdots,m)\\[0.1cm]
\qquad\left\{\begin{array}{l}
\max\limits_{{\mbox{\footnotesize\boldmath $y_i$}}}f_i(x,y_1,y_2,\cdots,y_m)\\[0.2cm]
\mbox{subject to:}\\[0.1cm]
\qquad g_i(x,y_1,y_2,\cdots,y_m)\le 0.
\end{array}\right.
\end{array}\right.
\end{equation}
这些跟规划相关的公式都来自于清华大学刘宝碇老师《不确定规划》的课件。以上的许多例子由清华大学的薛瑞尼同学编写。

\section{化学方程式}

使用\pkg{mhchem}的\csgo{ce}{化学式或方程式}能够让你很容易地表示出各种化学式和化学方程:

例如:
\begin{center}
	\ce{C6H5-CHO}\\ \ce{A\bond{~--}B\bond{~=}C\bond{-~-}D}\\ \ce{SO4^2- + Ba^2+ -> BaSO4 v}
\end{center}

复杂一点的方程式也不在话下,如\eqref{eq:chem}:
\begin{equation}\label{eq:chem}
	\ce{Zn^2+
		<=>[+ 2OH-][+ 2H+]
		$\underset{\text{amphoteres Hydroxid}}{\ce{Zn(OH)2 v}}$ <=>[+ 2OH-][+ 2H+]
		$\underset{\text{Hydroxozikat}}{\ce{[Zn(OH)4]^2-}}$
	}
\end{equation}

这个方程式嵌套在了\pkg{equation}环境中,可用\cs{eqlist}(\cs{listeq}的别名,作用相同)来编排到索引中。


\section{国际单位制(SI Unit)}

模板采用\pkg{siunitx}作为国际单位制支持宏包,以下是一些使用例子,这个包的文档写得非常不错,请在命令行里输入\texttt{texdoc siunitx}察看。
\begin{center}
	\num{.3e45}\\
	\num{1.654 x 2.34 x 3.430}\\
	\si{\kilogram\metre\per\second}\\    
	\SIlist{0.13;0.67;0.80}{\milli\metre}
\end{center}


\section{绘图}
\label{sec:draw}

本模板不预先装载任何绘图包(如 \pkg{pstricks,pgf} 等),完全由用户来决定。个人觉得 \pkg{pgf} 不错,不依赖于 Postscript。此外还有很多针对 \LaTeX{} 的GUI 作图工具,如 XFig(jFig), WinFig, Tpx, Ipe, Dia, Inkscape, LaTeXPiX,jPicEdt, jaxdraw 等等。

\section{插图}
\label{sec:graphs}

推荐《\LaTeXe\ 插图指南》。关于子图形的使用细节请参看 \pkg{subcaption} 宏包的说明文档。

\subsection{一个图形}
\label{sec:onefig}
一般图形都是处在浮动环境中。之所以称为浮动是指最终排版效果图形的位置不一定与源文
件中的位置对应\footnote{这是\LaTeX 的一个设计特性。},这也是刚使
用 \LaTeX{} 同学可能遇到的问题。如果要强制固定浮动图形的位置,请使用 \pkg{float} 宏包,
它提供了 \texttt{[H]} 参数,比如图~\ref{fig:xfig1}。
\begin{figure}[htb] % use float package if you want it here
	\centering
	\includegraphics[height=4cm]{bitlogo.pdf}
	\caption{北京理工大学大学校徽}
	\label{fig:xfig1}
\end{figure}

大学之道,在明明德,在亲民,在止于至善。知止而后有定;定而后能静;静而后能安;安
而后能虑;虑而后能得。物有本末,事有终始。知所先后,则近道矣。古之欲明明德于天
下者,先治其国;欲治其国者,先齐其家;欲齐其家者,先修其身;欲修其身者,先正其心;
欲正其心者,先诚其意;欲诚其意者,先致其知;致知在格物。物格而后知至;知至而后
意诚;意诚而后心正;心正而后身 修;身修而后家齐;家齐而后国治;国治而后天下
平。自天子以至于庶人,壹是皆以修身为本。其本乱而未治者 否矣。其所厚者薄,而其所
薄者厚,未之有也!

\hfill —— 《大学》


\subsection{多个图形}
\label{sec:multifig}

如果多个图形相互独立,并不共用一个图形计数器,那么用 \texttt{minipage} 或者\texttt{parbox} 就可以。否则,请参看
\begin{figure}[ht]
	\centering%
	\begin{subfigure}{3cm}
		\includegraphics[height=4cm]{bitlogo.pdf}
		\caption{第一个小图形}
	\end{subfigure}%
	\hspace{4em}%
	\begin{subfigure}{3cm}
		\includegraphics[height=3cm]{bitlogo.pdf}
		\caption{第二个小图形,注意这个图略矮些。subfigure中同一行的子图在顶端对齐。}
	\end{subfigure}
	\caption{包含子图形的大图形(subfigure示例)}
	\label{fig:big1-subfigure}
\end{figure}

如果要把编号的两个图形并排,那么小页就非常有用了。
\begin{figure}
	\begin{minipage}{0.48\textwidth}
		\centering
		\includegraphics[height=5cm]{bitlogo.pdf}
		\caption{并排第一个图}
		\label{fig:parallel1}
	\end{minipage}\hfill
	\begin{minipage}{0.48\textwidth}
		\centering
		\includegraphics[height=5cm]{bitlogo.pdf}
		\caption{并排第二个图}
		\label{fig:parallel2}
	\end{minipage}
\end{figure}

测试用途:theequation值为:\theequation ,thefigure值为:\thefigure ,thetable值为:\thetable



\begin{summary}
	
由于时间非常仓促,这个模板肯定存在不少问题,所以我需要大家帮助一起解决这
些问题。
下面是我自认为模板需要改进的地方。
1. 页边距设置上好像有问题,不同打印机打印效果不同;
2. 我只在ubuntu TEXLIVE2011 上做过测试,其他环境是否可行,尤为可知;
3. 本模板仅仅使用与硕士,不使用与博士、本科;
4. 本文档及其简单
这里是全文总结内容。

\end{summary} %% 全文总结


%%==============参考文献===================%%
%%==== 使用 BibTeX 编译========
% 注意:至少需要引用一篇参考文献,否则下面两行会引起编译错误
\nocite{*} %此处为测试,可以屏蔽掉
%\bibliographystyle{gbt-7714-2015-numerical}
%\bibliography{reference/test}
\printbitbib
%%============附录部分===========

\begin{appendix}
\addcontentsline{toc}{chapter}{附录}
\renewcommand\theequation{\Alph{chapter}--\arabic{equation}}  % 附录中编号形式是"A-1"的样子
\renewcommand\thefigure{\Alph{chapter}--\arabic{figure}}
\renewcommand\thetable{\Alph{chapter}--\arabic{table}}

\chapter{模板更新记录}
\label{chap:updatelog}

\textbf{2012年12月} 根据上海交通大学的模板由大眼睛,按照BIT的规范,构建了此模板。
\textbf{2017年6月} 根据最新研究生院要求重新设计,构建此模板
 % 此处为附录A,自己添加内容
\include{chapters/app2} % 此处为附录A,自己添加内容
\end{appendix}

% 发表文章目录
\backmatter
\include{chapters/pub}

% 致谢
%%==================================================
%% thanks.tex for SJTU Master Thesis
%% based on CASthesis
%% modified by wei.jianwen@gmail.com
%% version: 0.3a
%% Encoding: UTF-8
%% last update: Dec 5th, 2010
%%==================================================

\begin{thanks}

  {\color{red} 希望有牛人来维护咱们学校的模板。}

  感谢母校、感谢联盟!

  感谢高德纳!

  感谢提供交大硕士学位论文~\LaTeX~模板的同学!

  开源精神、奉献精神!

  \vskip 20mm
 {\zihao{-3}
  \hspace{80mm} {汪卫}

  \hspace{80mm} {2016年12月}

  \hspace{80mm} {于北京理工大学宇航大楼624} }
\end{thanks}


\end{document}
